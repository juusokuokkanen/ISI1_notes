%%%%%%%%  Document class  %%%%%%%%%%%%
\documentclass[10pt, twoside, a4paper]{book}

%%%%%%%%  Packages   %%%%%%%%%%%%%%%%
\usepackage{natbib}
%\usepackage{d:/laurim/biometria/lshort/src/lshort}
%\usepackage{numline}
%\usepackage{lineno}
\bibpunct{(}{)}{,}{a}{}{,}
\usepackage{amsmath,amssymb}
\usepackage{bm} 
\usepackage{verbatim}
\usepackage{enumerate}
\usepackage[T1]{fontenc}
\usepackage{hyperref}
\usepackage{color}

%\newenvironment{Rcode}[1]%
%{\scriptsize{\ttfamily{\bf The R-code for #1}} \\
%\ttfamily \noindent\ignorespaces }
%{\par\noindent%
%  \ignorespacesafterend}
  
\newenvironment{Rcode}[1]%
{\ttfamily{\bf The R-code for #1}
\scriptsize \\\par\verbatim}
{\endverbatim\par} 

\newenvironment{Rcode2}%
{\scriptsize \par\verbatim}%
{\endverbatim\par}

\usepackage{graphicx}
%\usepackage{subfigure}
%\usepackage[nolists]{endfloat}
\usepackage{amsthm}

\usepackage{times}
%\usepackage[T1]{fontenc}

\newcommand{\foldPath}{c:/laurim/tex/}

\newcommand{\beq}{\begin{equation*}}
\newcommand{\eeq}{\end{equation*}}

\DeclareMathOperator{\var}{var}
\DeclareMathOperator{\cov}{cov}
\DeclareMathOperator{\E}{E}
\DeclareMathOperator{\corr}{corr}
\DeclareMathOperator{\sd}{sd}

\theoremstyle{definition}

\newtheorem{example}{Example}[chapter]
\newtheorem{definition}{Definition}[chapter]

%\bibpunct{(}{)}{,}{a}{}{,}
\bibliographystyle{\foldPath saf}
%\bibliographystyle{alpha}
%\bibliographystyle{\foldPath wiley}
%\bibliographystyle{unsrt}

%%%%%%%%  Page Setup %%%%%%%%%%%%%%%%%%
%\topmargin      0pt
%\headheight     0pt 
%\headsep        0pt 
%\textheight   648pt 
%\oddsidemargin  0pt
%\textwidth    468pt 

%%%%%%%% Document %%%%%%%%%%%%%%%%%%%
\begin{document}
\pagenumbering{Roman}
\setlength{\baselineskip}{16pt}

%% Front matter
%\title{Forest biometrics with examples in R}
\title{Introduction to statistical inference 1}

\author{Lauri Meht\"atalo\\
	   University of Eastern Finland \\ School of Computing
	   }

\date{\normalsize \today}
\maketitle
\tableofcontents

%\section*{Preface}

%These are lecture notes for course Biometrics: R-statistics in Forest Sciences. In writing these notes, I have used several sources. In writing chapter 1, the main sources have been the second edition of the classical book ``Statistical Inference'' by George Casella and Roger L Berger\citep{Casellaandberger2002} and ``Methods for Forest Biometrics'' by Juha Lappi  \citep{Lappi1993}. In addition, I have used lecture notes of courses on mathematical statistics (by Eero Korpelainen) and statistical inference (by Jukka Nyblom). For chapters 2 and 3, the main sources have been the lecture notes of Jukka Nyblom on Regression analysis and Linear models, the book of Jose Pinheiro and Douglas Bates on Linear models with R \citep{Pinheiroandbates2000}, and the book ``Generalized, Linear and Mixed Models'' by CharlesMcCulloch and Shayle R Searle \citep{McCullochandsearle2001}. For the GLMs and GLMMs, I have used \citet{McCullochandsearle2001}, too. The last chapter on models systems is based mainly on book ``Econometric analysis'' by William H Greene \citep{Greene1997}. In addition, good sources of information have been the lecture notes of Annika Kangas on Forest Biometrics, the books of of Julian Faraway \citep{Faraway2004, Faraway2006}, Keith E Muller and Paul W Stewart \citep{Mullerandstewart2006}, and William N. Venables and Brian D. Ripley \citep{Venablesandripley2002}.   

\mainmatter
\pagenumbering{arabic}

\chapter{Preliminaries}
\section{Introduction}
\begin{itemize}
  \item Field of statistics builds on probability theory
  \begin{quotation}
  \textit{``You can, for example, never foretell what any one man will do, but
  you can say with precision what an average number will be up to. Individuals vary,
  but percentages remain constant. So says the statistician.''} - Sherlock Holmes
  \end{quotation}
  \item The paragraph includes the important ideas of the statistical model:
  \begin{itemize}
    \item The percentage $p$ = the model of the process or underlying
    population
    \item The behavior of individuals = data
  \end{itemize}
  \item Assuming a constat probability $p$ may be a too simplistic or naive
  assuption, and may be replaced by more realistic one where p is a function of
  the propabilities of the individual and the context where s/he is.
  \item We also need to specify a model, for the variability of the
  individuals around the $p$ to complete the model formulation.
  \begin{itemize}
    \item A crude summary if the variance-covariance matrix of the
    observations.
    \item A complete definition is done by specifying the joint distribution
    of all individuals.
  \end{itemize}
  \item We also may want to estimate how accurately we finally estimated $p$
  writing the available data
  \item The theoretical process that generates the data is called
  \begin{itemize}
    \item Statistical model  or (tilastollinen malli)
    \item Stochastic process  or (statistinen prosessi)
    \item Random process or (satunnaisprosessi)
  \end{itemize}
  \item The process is random/stochastic because the ``man'' do not behave
  exactly according to model.\footnote{This is what is done on this part of the
  course (ISI1).}
  \begin{itemize}
    \item Probability calculus and the theory of random variables provide
    tools to formulate and understand such models.
  \end{itemize}
  \item Once model has been formulated or specified (muotoiltu), observed data
  can be used to\footnote{This is what is done on second part of the course
  (ISI2)}
  \begin{itemize}
    \item estimate model parameters
    \item evaluate the model fit (mallin sopivuus)
    \item evaluate the inaccuracy related to the estimated model parameters
  \end{itemize}
  \item When talking about models, we can talk about
  \begin{itemize}
    \item True model (Tosi malli)
    \item Estimated model (Estimoitu malli)
    \item True model always stays the same, but as data used to formulate the
    estimated model gets larger, the estimated model gets closer to true model.
    \item See example R-scipt \verb#regsimu.R#
  \end{itemize}
\end{itemize}
\section{Set theory}
\begin{itemize}
	\item Consider a statistical experiment (e.g. rolling a dice, measuring the
	diameter of a tree, tossing a coin, measuring the photosynthetic activity in plant etc.)
\end{itemize}
\begin{definition}
All possible outcomes of a particular experiment (koe) form a set
(joukko) called sample space (otosavaruus), denoted by S. For example:

% \textbf{Examples} % TODO: Can this be done better?
\begin{itemize}
  \item[A] Toss of a coin; $ S = \{H, T\} $
  \item[B] Reaction time, Waiting time; $ S = [0, \infty) $
  \item[C] Exercise score of this course; $ S = \{0,1,2,\ldots,210\} $
  \item[D] Number of points (events) within fixed area; $ S = \{0,1,2,\ldots $
  \item[E] CO\textsubscript{2} uptake within 0.5 hours in fixed area plot; $ S =
  (-\infty, \infty) $
  \item[F] Waiting time up to one hour (in minutes); $ S = [0, 60) $
\end{itemize}

Sample space can be countable (numeroituva) or uncountable (ylinumeroituva). If
the elements of a sample space can be put into one-to-one correspondence with a
finite subset of integers, the space is countable. Otherwise, it is uncountable.
\end{definition}
\begin{itemize}
  \item Note: Examples A and C before are countable, the others are uncountable
  \item Note: If the waiting time in G are rounded to the
  minute / second / millisecond / microsecond, the sample space becomes
  countable.
\end{itemize}
\begin{definition}
An event (tapaus) is any collection of possible outcomes of an experiment,
meaning it is a subset of S. Event A is said to occur, if the outcome of the experiment is in set A.
\end{definition}
\begin{example}
Draw a card from standard deck.
$$S = \{ \heartsuit , \diamondsuit , \clubsuit, \spadesuit \}$$
One possible event is $A = \{ \heartsuit , \diamondsuit\}$.
Another possible event is $B = \{ \diamondsuit , \clubsuit, \spadesuit \}$. The
union (unioni) of the two events includes all elements of both
$$A \cup B = \{ \heartsuit , \diamondsuit , \clubsuit, \spadesuit \}$$
The intersection (leikkaus) includes elements that are common to both events
$$A \cap B = \{ \diamondsuit \}$$
The complement (komplementti) of a set includes al elements of $S$ that are not
included in $A$
$$A^c = \{ \clubsuit, \spadesuit \}$$
Events $A$ and $B$ are said to be disjoint (erillisi\"a), if 
$$A \cap B = \emptyset$$
Number of events $A_1$, $A_2$, $A_3$, \ldots are said to be pairwise disjoint,
if
$$A_i \cap A_j = \emptyset$$
for all pairs of i, j. In addition, if $\bigcup_{i=1}^{\infty}A_i = S$,
then $A_1$, $A_2$, $A_3$, \ldots defines a partition of the sample space.

\end{example}
\end{document}